\documentclass{beamer}
\usepackage[UTF8, noindent]{ctexcap}
\usetheme{warsaw}
\author{wenfengand}
\title{毕业设计(开题报告)}

\begin{document}

\begin{frame}
\frametitle{简介}
\framesubtitle{项目说明}
欢迎使用北京航空航天大学毕业设计论文毕业设计论文\LaTeX{}模板,
本模板由北航开源俱乐部(BHOSC)维护,根据北京航空航天大学教务处的
本科生毕业设计论文要求和研究生毕业设计论文要求来编写的。
\end{frame}

\begin{frame}{简介}{免责声明}

本模板为编写者依据北京航空航天大学研究生院及教务处出台的
《北京航空航天大学研究生撰写学位论文规定(2009年7月修订)》和
《本科生毕业设计(论文)撰写规范及要求》编写而成,
旨在方便北京航空航天大学毕业生撰写学位论文使用。

如前所述,本模板为北航开源俱乐部\LaTeX{}爱好者依据学校的要求规范编写,
研究生院及教务处只提供毕业论文的写作规范,目前并未提供官方\LaTeX{}模板,
也未授权第三方模板为官方模板,故此模板仅为论文规范的参考实现,
不保证格式能完全满足审查老师要求。任何由于使用本模板而引起的论文格式等问题,
以及造成的可能后果,均与本模板编写者无关。

任何组织或个人以本模板为基础进行修改、扩展而生成新模板,请严格遵守相关协议。
由于违反协议而引起的任何纠纷争端均与本模板编写者无关。

\end{frame}

\begin{frame}{简介}{列表测试}
中国在3000多年前就知道勾股数的概念,比古希腊更早一些。
《周髀算经》的记载:
\begin{itemize}
\item 公元前11世纪,商高答周公问:
\begin{quote}
勾广三,股修四,径隅五。
\end{quote}
\item 又载公元前7--6世纪陈子答荣方问,表述了勾股定理的一般形式:
\begin{quote}
若求邪至日者,以日下为勾,日高为股,勾股各自乘,并而开方除之,得邪至日。
\end{quote}
\end{itemize}
\end{frame}

\end{document}